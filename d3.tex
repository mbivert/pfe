\documentclass[a4paper]{article}

\usepackage{amsfonts}
\usepackage[francais,english]{babel}
\usepackage[T1]{fontenc}
\usepackage[]{fullpage}
\usepackage{graphicx}
\usepackage{hyperref}
\usepackage[utf8]{inputenc}
\usepackage{subfigure}

\makeatletter
\def\thickhrulefill{\leavevmode \leaders \hrule height 1pt\hfill \kern \z@}
\def\maketitle{%
  \null
  \thispagestyle{empty}%
  \vskip 1cm
  \begin{flushright}
        \normalfont\Large\@author
  \end{flushright}
  \vfil
  \hrule height 2pt
  \par
  \begin{center}
        \huge \strut \@title \par
  \end{center}
  \hrule height 2pt
  \begin{center}
  		Mars $2013$
  \end{center}
  \par
  \vfil
  \vfil
  \null
\begin{center}
\Huge{Placement constraints for a better QoS in clouds}
\end{center}
\begin{figure}[!ht]
	\centering
	\includegraphics[scale=.45]{imgs/cloud.png}
\end{figure}
\vfil
\begin{figure}[!ht]
	\centering
	\includegraphics[scale=.5]{imgs/polytech.png}
\end{figure}
\vfil
\begin{description}
	\item[Entreprise] Université de Nice-Sophia Antipolis
	\item[Lieu] Sophia-Antipolis, France
	\item[Responsable] Fabien Hermenier, équipe OASIS,
		\href{mailto:fabien.hermenier@unice.fr}{fabien.hermenier@unice.fr}
\end{description}
\cleardoublepage
}
\makeatother
\author{Mathieu Bivert, CSSR, \href{mailto:bivert@essi.fr}{bivert@essi.fr}}
\title{PFE : Rendu intérmédiaire $D_3$ (draft)}

\begin{document}
\maketitle

\selectlanguage{francais}
	\begin{abstract}
		L'ajout d'un type représentant les hyperviseurs dans $D_2$ permet
		l'implémentation de nouvelles contraintes dans BtrPlace. Ce
		document en présente quelques unes. Le lecteur est renvoyé à
		la première section de $D_2$ pour des définitions précises du
		vocabulaire employés ici.
	\end{abstract}

\selectlanguage{english}
	\begin{abstract}
		The addition of a type to represent the available hypervisors
		for a node in $D_2$ allows to implement new constraints in
		BtrPlace. This documents presents some of those possible
		constraints. The reader may refer to $D_2$ to find definitions
		of the vocabulary being used here.
	\end{abstract}

\selectlanguage{francais}

\tableofcontents
\newpage

\section{Motivations}
BtrPlace est un logiciel permettant de trouver efficacement une solution
à un problème de placement de machines virtuelles sur des serveurs
physiques. Pour ce faire, l'utilisateur utilise des contraintes fournies
par BtrPlace. Celles-ci permettent de rendre compte de certaines
limitations (eg. ressources mémoire/cpu/réseau), d'une topologie
particulière des composants d'une application (eg. réplicat de
serveurs pour augmenter la disponibilité), etc.

Dans $D_2$, le modèle théorique de BtrPlace est étendu afin de permettre
le typage des nœuds et des plateformes. Les contraintes qui en découlent
portent sur la compatibilité entre machines virtuelles et nœuds, sur
la capacité à un serveur physique de faire tourner un hyperviseur donné,
ou encore sur le nombre de machines virtuelles d'un type donné pouvant
tourner sur un hyperviseur.

\section{Interaction avec le modèle de $D_2$}
\subsection{Modèlisation}
\subsection{Implémentation}
Contraintes de placement, donc implémenter PlacementConstraint
\subsection{Validation}

\section{Exemples de contraintes}
\subsection{Nombre maximum et minimum de machines virtuelles}

\end{document}
