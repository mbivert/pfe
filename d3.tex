\documentclass[a4paper]{article}

\usepackage{amsfonts}
\usepackage[francais,english]{babel}
\usepackage[T1]{fontenc}
\usepackage[]{fullpage}
\usepackage{graphicx}
\usepackage{hyperref}
\usepackage[utf8]{inputenc}
\usepackage{subfigure}

\makeatletter
\def\thickhrulefill{\leavevmode \leaders \hrule height 1pt\hfill \kern \z@}
\def\maketitle{%
  \null
  \thispagestyle{empty}%
  \vskip 1cm
  \begin{flushright}
        \normalfont\Large\@author
  \end{flushright}
  \vfil
  \hrule height 2pt
  \par
  \begin{center}
        \huge \strut \@title \par
  \end{center}
  \hrule height 2pt
  \begin{center}
  		Mars $2013$
  \end{center}
  \par
  \vfil
  \vfil
  \null
\begin{center}
\Huge{Placement constraints for a better QoS in clouds}
\end{center}
\begin{figure}[!ht]
	\centering
	\includegraphics[scale=.45]{imgs/cloud.png}
\end{figure}
\vfil
\begin{figure}[!ht]
	\centering
	\includegraphics[scale=.5]{imgs/polytech.png}
\end{figure}
\vfil
\begin{description}
	\item[Entreprise] Université de Nice-Sophia Antipolis
	\item[Lieu] Sophia-Antipolis, France
	\item[Responsable] Fabien Hermenier, équipe OASIS,
		\href{mailto:fabien.hermenier@unice.fr}{fabien.hermenier@unice.fr}
\end{description}
\cleardoublepage
}
\makeatother
\author{Mathieu Bivert, CSSR, \href{mailto:bivert@essi.fr}{bivert@essi.fr}}
\title{PFE : Rendu intérmédiaire $D_3$ (draft)}

\begin{document}
\maketitle

\selectlanguage{francais}
	\begin{abstract}
		L'ajout d'un type représentant les hyperviseurs dans $D_2$ permet
		l'implémentation de nouvelles contraintes dans BtrPlace. Ce
		document en présente quelques unes. Le lecteur est renvoyé à
		la première section de $D_2$ pour des définitions précises du
		vocabulaire employés ici.
	\end{abstract}

\selectlanguage{english}
	\begin{abstract}
		The addition of a type to represent the available hypervisors
		for a node in $D_2$ allows to implement new constraints in
		BtrPlace. This documents presents some of those possible
		constraints. The reader may refer to $D_2$ to find definitions
		of the vocabulary being used here.
	\end{abstract}

\selectlanguage{francais}

\tableofcontents
\newpage

\section{Motivations}
BtrPlace est un logiciel permettant de trouver efficacement une solution
à un problème de placement de machines virtuelles sur des serveurs
physiques. Pour ce faire, l'utilisateur utilise des contraintes fournies
par BtrPlace. Celles-ci permettent de rendre compte de certaines
limitations (eg. ressources mémoire/cpu/réseau), d'une topologie
particulière des composants d'une application (eg. réplicat de
serveurs pour augmenter la disponibilité), etc.

Dans $D_2$, le modèle théorique de BtrPlace est étendu afin de permettre
le typage des nœuds et des plateformes. Les contraintes qui en découlent
portent sur la compatibilité entre machines virtuelles et nœuds, sur
la capacité à un serveur physique de faire tourner un hyperviseur donné,
ou encore sur le nombre de machines virtuelles d'un type donné pouvant
tourner sur un hyperviseur.

\section{Implémentation du modèle de $D_2$}
Deux premières contraintes \textit{Platform} et \textit{TypedPlatform}
permettent respectivement de changer le type d'un ensemble de nœud, et
de s'assurer que les VMs d'un type donné sont bien placés sur un nœud
du même type.

Le lecteur est renvoyé au $D_2$ pour une description plus précise.

\section{Nouvelles contraintes de placement}
\subsection{Motivations}
\subsubsection{Licences}
Comme mentionné dans les autres documents, les licences des hyperviseurs
limitent les ressources pouvant être utilisées. Par exemple,
VMWare\footnote{\url{https://www.vmware.com/support/licensing/per-vm/}}
limite le nombre de machines virtuelles pouvant tourner sur un même nœud;
Xen limite la mémoire utilisable pour un hyperviseur, ou encore le nombre
d'interfaces réseaux pouvant être connectées simultanément, voire la puissance
CPU.

Des contraintes portant sur le type ajoutée peuvent rendre compte de ces
limitations.
\subsubsection{QoS}
Dans le cas d'un déploiement réel, on peut chercher à optimiser le temps
d'indisponibilité d'un service. Par exemple, supposons qu'un nœud sous un
hyperviseur Xen tombe en panne : il est impératif que les VMs situées sur
celui-ci soient migrées au plus vite. Avoir sous la main un nœud « vide »
tournant sous Xen permet d'éviter d'avoir à redémarrer un nœud.

Une contrainte s'assurant qu'il existe au moins un certain nombre de nœud
tournant sous un hyperviseur donné permet de palier à ce problème. On
pourrait raffiner la contrainte, et s'assurer que pour un type, il existe
au moins un nœud capable de recevoir telle ou telle VM en cas de problèmes.

\subsection{Validation}
De même que pour la contrainte du cas général, ces contraintes ne sont
pas validées par manque de temps. Il faudrait définir des cas d'utilisation
précis avec l'encadrant, puis implémenter des tests unitaires les mettant
en œuvre.

\section{Exemples de contraintes}
L'implémentations est réduite au strict minimum, c'est-à-dire à l'injection
de contraintes dans choco et au test de leur satisfiabilité.
\subsection{Nombre maximum de machines virtuelles}
La contrainte \textit{MaxVM}\footnote{https://github.com/Heaumer/pfe/blob/master/entropy-fh/src/main/java/entropy/vjob/MaxVM.java}
implémente le cas décrit dans la section précèdente. Son constructeur prends
en argument un ensemble de nœud, le type sur lequel porte la limitation, et
enfin le nombre maximum de VMs pour ce type. Pour chaque nœud, on regarde
si une action de retypage est prévue. Si c'est le cas, et que le nouveau
type correspond à celui spécifié par l'utilisateur, on s'assure que le nombre
de VMs sur ce nœud est inférieur à la borne.

\subsection{Nombre minimum de nœud d'un type donné}
La contrainte \textit{MinPlatform}\footnote{https://github.com/Heaumer/pfe/blob/master/entropy-fh/src/main/java/entropy/vjob/MinPlatform.java}
spécifie que pour un type donné, au moins \textit{nType} nœuds tournent sous
un hyperviseur \textit{type}. Pour ce faire, une variable est incrémenté à
chaque fois qu'un nœud du type requis est trouvé, puis, on renseigne choco
sur le fait que cette variable doit être inférieur à la borne donnée par
l'utilisateur.

\end{document}
