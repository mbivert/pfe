On commence par définir le vocabulaire:
\begin{description}
	\item[Type] entier $t$ associé à chaque système de virtualisation;
	\item[VM] machine virtuelle, notée $v \in \mathcal V$, à laquelle
		est associée un type fixe $T(v)$ et une place $P(v)$;
	\item[Nœud] serveur physique, noté $n \in \mathcal N$,doté d'un
		type courant $T(n)$ et d'un ensemble de types possibles
		$\mathcal{T}_n$;
\end{description}

% On note $N_n$ le nombre de nœud; $N_v$ le nombre de VMs (c'est-à-dire
% respectivement les cardinaux de $\mathcal{N}$ et de $\mathcal{V}$).

La fonction $T$ associe à une VM ou un Nœud son type; la fonction $P$
associe à une VM un nœud.

Le placement est alors satisfait ssi chaque VM est bien placée sur
un nœud de même type, ie.:
\[
	(\forall v \in \mathcal V), (\exists n \in \mathcal N), P(v) = n
		\Rightarrow T(n) = T(v)	
\]

L'un des cas possible pour satisfaire cette condition et de changer le
type courant d'une machine virtuelle. Une action de déploiement doit
alors être mise en place.

La modélisation d'actions de reconfiguration est définie dans ~\cite{herm2012}.
Elles sont réalisées à l'aide de \textit{slices}, qui correspondent à
une durée finie pendant un processus de reconfiguration, durant laquelle
des ressources sont utilisées.
On distingue plusieurs types de slices:
\begin{description}
	\item[consuming slice], $c \in \mathcal C$, où les ressources sont
		utilisées au début de la reconfiguration;
	\item[demanding slice], $d \in \mathcal D$, où les ressources sont
		utilisées à la fin de la reconfiguration;
	\item[middle slice], $m \in \mathcal M$, où les ressources sont utilisées
		entre le début et la fin du processus (XXX ambigüe, notation).
\end{description}

L'opération de déploiement peut s'exprimer en fonction de ces slices:
\begin{enumerate}
	\item l'état initial (c-slice) lors de la reconfiguration contient les
		VMs de l'ancien type devant être déplacées sur un autre serveur ;
	\item l'état intermédiaire (m-slice) représente le changement de type,
		c'est-à-dire le changement de système de virtualisation. Il
		peut être vu comme consommant toutes les ressources disponibles
		sur le nœud;
	\item l'état final (d-slice) où les VMs du nouveau type sont migrées sur
		le nœud.
\end{enumerate}

