\documentclass[a4paper]{article}

\usepackage{amsfonts}
\usepackage[francais,english]{babel}
\usepackage[T1]{fontenc}
\usepackage[]{fullpage}
\usepackage{graphicx}
\usepackage{hyperref}
\usepackage[utf8]{inputenc}
\usepackage{subfigure}

\makeatletter
\def\thickhrulefill{\leavevmode \leaders \hrule height 1pt\hfill \kern \z@}
\def\maketitle{%
  \null
  \thispagestyle{empty}%
  \vskip 1cm
  \begin{flushright}
        \normalfont\Large\@author
  \end{flushright}
  \vfil
  \hrule height 2pt
  \par
  \begin{center}
        \huge \strut \@title \par
  \end{center}
  \hrule height 2pt
  \begin{center}
  		$7$ Mars $2013$
  \end{center}
  \par
  \vfil
  \vfil
  \null
\begin{center}
\Huge{Placement constraints for a better QoS in clouds}
\end{center}
\begin{figure}[!ht]
	\centering
	\includegraphics[scale=.45]{imgs/cloud.png}
\end{figure}
\vfil
\begin{figure}[!ht]
	\centering
	\includegraphics[scale=.5]{imgs/polytech.png}
\end{figure}
\vfil
\begin{description}
	\item[Coût du livrable] $38$ heures
	\item[Budget total du projet] $304$ heures
	\item[Entreprise] Université de Nice-Sophia Antipolis
	\item[Lieu] Sophia-Antipolis, France
	\item[Responsable] Fabien Hermenier, équipe OASIS,
		\href{mailto:fabien.hermenier@unice.fr}{fabien.hermenier@unice.fr}
\end{description}
\cleardoublepage
}
\makeatother
\author{Mathieu Bivert, CSSR, \href{mailto:bivert@essi.fr}{bivert@essi.fr}}
\title{PFE : Rapport de management ($D_4$)}

\begin{document}
\maketitle
\tableofcontents
\newpage
\section{Description du projet}
On renvoye le lecteur à la section \textit{Description du projet} du DOW
pour description plus précise de l'environnement, du vocabulaire et des
besoins de l'utilisateur.

BtrPlace est un logiciel développé à l'INRIA par Fabien Hermenier qui place
efficacement un ensemble de machines virtuelles sur un ensemble de serveurs
physiques à l'aide de contraintes. Celles-ci sont principalement de deux
natures:
\begin{enumerate}
	\item les contraintes spécifiées par l'utilisateur, qui répondent
		à des cas d'utilisations particuliers, par exemple
		avoir à disposition un serveur sous un hyperviseur donné pour y migrer
		des VMs en cas de problèmes sur un autre serveur;
	\item les contraintes imposées par les ressources disponibles, portant
		par exemple sur la mémoire, la puissance de calcul disponibles sur
		un nœud.
\end{enumerate}

Dans le cadre de ce projet, on a travaillé en deux temps:
\begin{enumerate}
	\item ajouter au modèle théorique derrière BtrPlace la possibilité de
		\textit{typer} les nœuds et les VMs pour rendre compte de la présence
		de multiples hyperviseurs (Xen, VMWare, etc.);
	\item concevoir et implémenter des \textit{contraintes} mettant en œuvre
		ce typage.
\end{enumerate}

Pour ce faire, le travail a suivi une progression incrémentale:
\begin{enumerate}
	\item on a commencé par chercher un cas simple du typage, et on
		a implémenté une contrainte le décrivant;
	\item dans un second temps, on a exprimé le typage dans un cas plus
		général, d'abords mathématiquement, puis, toujours d'une façon
		abstraite, sous forme de contraintes;
	\item enfin, on implémente des contraintes dans BtrPlace avec Choco.
\end{enumerate}

\section{Synthèse des résultats obtenus}
\subsection{Support du typage}
\subsubsection{Cas particulier}
On a ajouté une contrainte \textit{Platform} permettant d'associer à
des nœuds un nouvel type, c'est-à-dire en pratique, d'effectuer une
action de retypage sur un ensemble de nœuds. Cette dernière s'assure que:
\begin{enumerate}
	\item les VMs sont parties avant le redémarrage du nœud;
	\item les VMs pouvant migrer sur ce nœud ne le feront qu'après le
		redémarrage de celui-ci.
\end{enumerate}
\subsubsection{Cas général}
On associe à chaque nœud un vecteur contenant un élément pour
chaque hyperviseur possible. Sémantiquement, les éléments de ce
vecteur représentent le nombre de machines virtuelles d'un type donné
tournant sur ce nœud. Une valeur nulle indique que l'hyperviseur
n'est pas utilisé; si le vecteur ne contient que des zéros, alors
le nœud est considéré comme éteint.

Il s'en suit qu'une seule composante du vecteur peut-être non nulle :
un nœud ne peut pas faire tourner deux hyperviseurs simultanément.

\subsection{Nouvelles contraintes}
En plus de la contrainte du cas particulier, deux nouvelles
contraintes peuvent être implémentée. La première, \textit{MaxVM}
permet de répondre aux limitations induites par les licences des logiciels de
virtualisation\footnote{\url{https://www.vmware.com/support/licensing/per-vm/}},
en fixant une limite au nombre de machines virtuelles
tournant sur un nœud. Afin de s'assurer de
la présence de certaines plateformes pour faciler les opérations de
déploiement future, la contrainte \textit{MinPlatform} s'assure qu'il
existe au moins un nombre donné de nœuds faisant tourner un hyperviseur.

\section{Implication des personnes}
\subsection{Fabien Hermenier}
Suite à une mauvaise compréhension de éléments du DOW, en partie due à
une rédaction un peu hâtive de celui-ci, suite à une assignation tardive
au projet, les heures passées avec l'encadrant n'ont pas été intégrées
dans le DOW.

Quelques entretiens avec Fabien à l'INRIA m'ont permis de me débloquer
sur certains points, mais surtout d'établir des lignes directrices et
des méthodes d'avancement. Enfin, une correspondance assez régulière
par emails m'a permis d'avoir des retours sur le travail effectué, ce
qui m'a fortement aidé à progresser.
\subsection{Mathieu Bivert}
Avec Fabien, nous avons définis un plan d'action pour effectuer le travail
demandé, puis nous avons:
\begin{itemize}
	\item mis au clair les besoins concernant l'augmentation du modèle;
	\item proposé une modélisation mathématiques du typage des nœuds et
		des VMs;
	\item enfin, proposé une implémentation en deux temps du typage.
\end{itemize}

\section{Synthèses des livraisons}
Les livraisons sont effectuées pour mercredi $7$ mars $2013$, ce qui n'est
pas conforme au planning prévu. L'un des problèmes rencontrés
repose sur l'approche incrémentale utilisée. Bien que celle-ci permette
de progresser facilement, elle a tendance à lier fortement les différentes
composantes du projet. L'interdépendance entre contraintes et implémentation
du modèle limite la possibilité de rendre au temps fixé dans le DOW les
différents livrables.

Pour la même raison, la structure des livrables $D_2$ et $D_3$ n'est pas
optimale : en effet, la majeure partie de l'implémentation du modèle décrit
dans $D_2$ est à placer dans les contraintes présentes dans $D_3$. Ces
problèmes n'avaient pas été envisagés dans le DOW. Une solution aurait été
de découper les rapports différement, par exemple décrire la partie théorique
dans $D_2$ et l'implémentation dans $D_3$, ou encore fusionner les deux
documents. Dans les deux cas, le DOW n'aurait pas été respecté.

Enfin, des incohérences au sein du DOW rendent son interprétation difficile.

\section{Suivi budgétaire}
\subsection{Consommation du budget}
Le temps de rédaction des rapports et pour la gestion du management du projet
ont été surévalués : il aurait été plus judicieux de passer plus de temps
sur l'implémentation du code.
\subsection{Synthèse}
Globalement, le budget fixé a été respecté. Le projet était peut-être un
peu trop ambitieux au vu du temps difficile à prévoir pour les cours et
évènements para-scolaires.
\section{Suivi des lots}
Cette section ne correspond pas à celle annoncée dans le DOW. En effet, des
incohérences sur le découpage de projet rendent la partie difficilement
utilisable. On se propose ici, de décrire les rendus plus que les lots
prévus. La description du travail du DOW est toujours d'actualité.

Une quinzaine d'heures de travail n'ont pas été consommée par rapport
au budget de $304$ heures (environ $5$% du temps total) suite à une
mauvaise compréhension de la date de fin, qui s'est avérer être deux
jours plus tôt.

\begin{figure}[!ht]
	\centering
	\includegraphics[scale=.5]{imgs/pie.png}
\end{figure}

Le code du cas particulier et le modèle du cas général ont été validés
par l'encadrant.

\subsection{Management du projet}
La planification du projet (DOW) ainsi que le rapport ci-présent sont
rendus en temps, et ont pris environ un tiers du temps total passé sur le
projet ($30\%$).
\subsection{Ajout du typage}
L'extension du modèle théorique a été relativement rapide, et a pris
dans les $10\%$ du temps passé à travailler.
\subsection{Implémentation des contraintes}
L'implémentation a pris la plus grosse partie du temps (environ $60\%$),
et n'est malheureusement pas complète : la validation des contraintes
n'est pas toujours effectuée, et l'intégration du cas général pour le typage
est incomplète.

La section suivante illustre plus en détails les difficultées rencontrées
pour cette partie, et propose une solution.
\section{Synthèse et retour d'expérience}
En ne prenant en compte que le projet, la gestion du temps était
loin d'être optimale. Bien que la partie théorique soit assez facile
à comprendre, sa mise en œuvre dans BtrPlace est bien plus difficile.
La complexité de l'implémentation est à mon avis l'un des facteurs
majeurs justifiant le temps passé à implémenter le modèle.

Une solution envisageable à ce problème et peu coûteuse en temps, serait 
de décrire BtrPlace à l'aide de deux cartes:
\begin{itemize}
	\item une première de « haut niveau », présentant la structure
		du logiciel, accompagnée d'exemples de bases et de quelques
		options, permettant à un novice de savoir où il est et où il va;
	\item une deuxième de « moyen niveau », qui explique comment le
		modèle est généré, et comment coder quelques contraintes.
\end{itemize}
Une API de programation est déjà présente, et fait déjà office de
carte « bas niveau ». Celle-ci n'est cependant pas suffisante pour
prendre en main efficacement le logiciel.

À l'heure actuelle, les ajouts en terme de code sont incomplets, mais:
\begin{enumerate}
	\item l'implémentation du typage telle que décrite plus haut
		permet de résoudre un autre projet proposé par Fabien
		Hermenier, à savoir la gestion de la limitation sur le nombre
		de VMs pour un hyperviseur donné pour des questions de
		licence;
	\item le modèle théorique peut-être facilement étendu pour gérer
		d'autres limitations au niveau des ressources, en fonction
		du type de l'hyperviseur. En effet, les licences des systèmes
		de virtualisation imposent généralement des limites sur la
		quantité de RAM utilisable, sur la puissance CPU ou encore
		limite le nombre d'interface réseau, la quantité de RAM, etc.
\end{enumerate}

Au niveau des compétences, le projet m'a permis de revisiter
certains éléments, tels que :
\begin{itemize}
	\item Java;
	\item les tests unitaires pour la validation;
	\item git comme gestionnaire de version;
	\item la modélisation mathématiques
	\item programmation par contraintes
\end{itemize}
et d'en acquérir de nouvelles, comme:
\begin{itemize}
	\item IDE (IntelliJ);
	\item outil de déploiement (Maven);
	\item framework Choco;
	\item gestionnaire de ressources;
	\item problèmes combinatoires
\end{itemize}

Du point de vue du management, j'ai appris qu'il me faut mieux gérer
mon temps, notamment en essayant de mieux évaluer les évènements exogènes.

\end{document}
