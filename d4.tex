\documentclass[a4paper]{article}

\usepackage{amsfonts}
\usepackage[francais,english]{babel}
\usepackage[T1]{fontenc}
\usepackage[]{fullpage}
\usepackage{graphicx}
\usepackage{hyperref}
\usepackage[utf8]{inputenc}
\usepackage{subfigure}

\makeatletter
\def\thickhrulefill{\leavevmode \leaders \hrule height 1pt\hfill \kern \z@}
\def\maketitle{%
  \null
  \thispagestyle{empty}%
  \vskip 1cm
  \begin{flushright}
        \normalfont\Large\@author
  \end{flushright}
  \vfil
  \hrule height 2pt
  \par
  \begin{center}
        \huge \strut \@title \par
  \end{center}
  \hrule height 2pt
  \begin{center}
  		Mars $2013$
  \end{center}
  \par
  \vfil
  \vfil
  \null
\begin{center}
\Huge{Placement constraints for a better QoS in clouds}
\end{center}
\begin{figure}[!ht]
	\centering
	\includegraphics[scale=.45]{imgs/cloud.png}
\end{figure}
\vfil
\begin{figure}[!ht]
	\centering
	\includegraphics[scale=.5]{imgs/polytech.png}
\end{figure}
\vfil
\begin{description}
	\item[Coût du livrable] $38$ heures
	\item[Budget total du projet] $304$ heures
	\item[Entreprise] Université de Nice-Sophia Antipolis
	\item[Lieu] Sophia-Antipolis, France
	\item[Responsable] Fabien Hermenier, équipe OASIS,
		\href{mailto:fabien.hermenier@unice.fr}{fabien.hermenier@unice.fr}
\end{description}
\cleardoublepage
}
\makeatother
\author{Mathieu Bivert, CSSR, \href{mailto:bivert@essi.fr}{bivert@essi.fr}}
\title{PFE : Rapport de management ($D_4$) (draft)}

\begin{document}
\maketitle
\tableofcontents
\newpage
\section{Description du projet}
On renvoye le lecteur à la section \textit{Description du projet} du DOW
pour description plus précise de l'environnement, du vocabulaire et des
besoins de l'utilisateur.

BtrPlace est un logiciel développé à l'INRIA par Fabien Hermenier qui place
efficacement un ensemble de machines virtuelles sur un ensemble de serveurs
physiques à l'aide de contraintes. Celles-ci sont principalement de deux
natures:
\begin{enumerate}
	\item les contraintes spécifiées par l'utilisateur, qui répondent
		généralement à des cas d'utilisations particuliers;
	\item les contraintes imposées par les ressources disponibles, portant
		par exemple sur la mémoire, la puissance de calcul disponibles sur
		un nœud.
\end{enumerate}

Dans le cadre de ce projet, on se propose de travailler en deux temps:
\begin{enumerate}
	\item ajouter au modèle théorique sous-jacent là possibilité de
		\textit{typer} les nœuds et les VMs pour rendre compte de la présence
		de multiples hyperviseurs (Xen, VMWare, etc.);
	\item concevoir et implémenter des \textit{contraintes} mettant en œuvre
		ce typage.
\end{enumerate}

Pour ce faire, le travail a suivi une progression incrémentale:
\begin{enumerate}
	\item on a commencé par chercher un cas simple du typage, et on
		implémente une contrainte le décrivant;
	\item dans un premier temps, on exprime le typage dans un cas plus
		général, d'abords mathématiquement, puis, toujours d'une façon
		abstraite, sous forme de contrainte;
	\item enfin, on implémente des contraintes mettant en œuvre ceci.
\end{enumerate}

\section{Synthèse des résultats obtenus}
\subsection{Support du typage}
\subsubsection{Cas particulier}
On ajoute une contrainte permettant d'associer à des nœuds un
nouvel type, c'est-à-dire en pratique, d'effectuer une action de
retypage sur un ensemble de nœuds. Cette dernière se décompose en:
\begin{enumerate}
	\item migrer les éventuelles VMs tournant sur ce nœuds;
	\item éteindre le nœud;
	\item le redémarrer sous un nouvel hyperviseur.
\end{enumerate}
\subsubsection{Cas général}
On associe à chaque nœud un vecteur contenant un élément pour
chaque hyperviseur possible. Sémantiquement, les éléments de ce
vecteur représentent le nombre de machines virtuelles d'un type donné
tournant sur ce nœud. Une valeur nulle indique que l'hyperviseur
n'est pas utilisé; si le vecteur ne contient que des zéros, alors
le nœud est considéré comme éteint.

Il s'en suit qu'une seule composante du vecteur peut-être non nulle :
un nœud ne peut pas faire tourner deux hyperviseurs simultanément.

\subsection{Nouvelles contraintes}
En plus de la contrainte du cas particulier, deux nouvelles
contraintes peuvent être implémentée. La première permet de répondre
aux limitations induites par les licences des logiciels de
virtualisation, en fixant une limite au nombre de machines virtuelles
tournant sur un nœud. Une variante de celle-ci est de restreindre
le nombre de nœuds tournant de façon simultanée sous un même
hyperviseur. Enfin, afin de s'assurer de la présence de certaines
plateformes pour faciler les opérations de déploiement future, on
peut aussi s'assurer qu'il existe au moins un nombre donné de nœuds
faisant tourner un hyperviseur.

\section{Implication des personnes}
XXX
\subsection{Fabien Hermenier}
\subsection{Mathieu Bivert}

\section{Synthèses des livraisons}
Les livraisons sont effectués pour mercredi $7$ mars $2013$, ce qui n'est
pas conforme au planning prévu. L'un des problèmes majeurs recontrés
repose sur l'approche incrémentale utilisée. Bien que celle-ci permette
de progresser facilement, elle a tendance à lier fortement les différentes
composantes du projet. L'interdépendance entre les éléments des lots complique
la possibilité de rendre au temps fixé dans le DOW les différents livrables.

Pour la même raison, la structure des livrables $D_2$ et $D_3$ n'est pas
optimale : en effet, la majeure partie de l'implémentation du modèle décrit
dans $D_2$ est à placer dans les contraintes présentes dans $D_3$. Ces
problèmes n'avaient pas été envisagés dans le DOW.

\section{Suivi budgétaire}
\subsection{Consommation du budget}
Le temps de rédaction des rapports et pour la gestion du management du projet
ont été surévalués : il aurait été plus judicieux de passer plus de temps
sur l'implémentation du code.
\subsection{Synthèse}
Globalement, le budget fixé a été respecté. Le projet était peut-être un
peu trop ambitieux au vu du temps difficile à prévoir pour les cours et
évènements para-scolaires.
\section{Suivi des lots}
XXX
\section{Synthèse et retour d'expérience}
En ne prenant en compte que le projet, la gestion du temps était
loin d'être optimale. Bien que la partie théorique soit assez facile
à comprendre, sa mise en œuvre dans BtrPlace est bien plus difficile.
La complexité de l'implémentation est à mon avis l'un des facteurs
majeurs justifiant le temps passé à implémenter le modèle.
Une solution à ce problème de complexité serait de ré-implémenter
BtrPlace dans un langage plus adapté pour la programmation par
contrainte (Prolog, Lisp, etc.), mais cela nécessite d'étudier le
coût d'une telle ré-écriture. En effet, la nouvelle solution peut-être
plus facile à comprendre et à étendre, mais elle peut aussi:
\begin{enumerate}
	\item être moins performante et plus difficilement optimisable;
	\item nécessiter un temps d'implémentation pas forcément rentable
		par rapport au temps pouvant être mis à étendre le code actuel
		par une personne en ayant une conaissance approfondie.
\end{enumerate}

À l'heure actuelle, les ajouts en terme de code sont incomplets, mais:
\begin{enumerate}
	\item l'implémentation du typage telle que décrite plus haut
		permet de résoudre un autre projet proposé par Fabien
		Hermenier, à savoir la gestion de la limitation sur le nombre
		de VMs pour un hyperviseur donné pour des questions de
		licence;
	\item le modèle théorique peut-être facilement étendu pour gérer
		d'autres limitations au niveau des ressources, en fonction
		du type de l'hyperviseur. En effet, les licences des systèmes
		de virtualisation imposent généralement des limites sur la
		quantité de RAM utilisable, sur la puissance CPU ou encore
		limite le nombre d'interface réseau, la quantité de RAM, etc.
\end{enumerate}

\end{document}
