\documentclass[a4paper]{article}

\usepackage[francais,english]{babel}
\usepackage[utf8]{inputenc}
\usepackage[T1]{fontenc}
\usepackage{graphicx}
\usepackage{hyperref}

\usepackage[]{fullpage}

\makeatletter
\def\thickhrulefill{\leavevmode \leaders \hrule height 1pt\hfill \kern \z@}
\def\maketitle{%
  \null
  \thispagestyle{empty}%
  \vskip 1cm
  \begin{flushright}
        \normalfont\Large\@author
  \end{flushright}
  \vfil
  \hrule height 2pt
  \par
  \begin{center}
        \huge \strut \@title \par
  \end{center}
  \hrule height 2pt
  \par
  \vfil
  \vfil
  \null
\begin{center}
\Huge{Placement constraints for a better QoS in clouds}
\end{center}
\begin{figure}[!ht]
        \centering
        \includegraphics[scale=.45]{imgs/cloud.png}
\end{figure}
\vfil
\begin{figure}[!ht]
        \centering
        \begin{minipage}[c]{.46\linewidth}
                \includegraphics[scale=.3]{imgs/inria.png}
        \end{minipage}
        \begin{minipage}[c]{.2\linewidth}
                \includegraphics[scale=.5]{imgs/polytech.png}
        \end{minipage}
\end{figure}
\vfil
\begin{description}
        \item[Entreprise] INRIA
        \item[Lieu] Sophia-Antipolis, France
        \item[Responsable] Fabien Hermenier, équipe OASIS,
        	\href{mailto:fabien.hermenier@unice.fr}{fabien.hermenier@unice.fr}
\end{description}
  \cleardoublepage
  }
\makeatother
\author{Mathieu Bivert, CSSR, \href{mailto:bivert@essi.fr}{bivert@essi.fr}}
\title{PFE: Cahier des charges}

\begin{document}
\maketitle

\selectlanguage{francais}
\begin{abstract}
Blablabla français
\end{abstract}

\selectlanguage{english}
\begin{abstract}
Blablabla english
\end{abstract}

\tableofcontents
\newpage
\section{Description du projet}
\subsection{Contexte de travail}
Les clouds (Amazon EC2, Microsoft Azure, etc.) proposent à leurs clients
un accès à une infrastructure informatique. Une entreprise peut ainsi
déléguer une grande partie de l'administration et de son matériel sur
un cloud, ce qui lui permet d'avoir une garantie quant-à la qualité du
service fournis, tout en réduisant les coûts.

Afin de rentabiliser au mieux le matériel, les fournisseurs de clouds
utilisent généralement la virtualisation : un même serveur physique
peut ainsi héberger plusieurs systèmes virtuels, fournissant des services
quelconques (emails, serveurs web, accès à un OS complet, etc.).

\subsection{Motivations}
La question de la répartition des machines virtuelles sur les machines
physiques se pose alors pour des raisons diverses et variées:
\begin{description}
	\item[maintenance] un serveur physique peut tomber en panne, ou
		nécessiter une réparation, auquel cas les programmes
		tournant dessus doivent être migré ailleurs, afin de
		garantir au client une certaine qualité de service (QoS);
	\item[sécurité] il peut s'avérer risquer pour un programme d'un
		client traitant des données sensibles (eg. données bancaires)
		de se retrouver au même endroit qu'un programme d'un
		autre client;
	\item[économie d'énergie] où il peut être avantageux de réduire
		le nombre de serveurs physiques allumés, pour maximiser
		le rendement des autres machines physiques du cloud;
	\item[QoS] où, à l'inverse de l'économie d'énergie, il est bon
		de garder des ressources supplémentaires disponibles immédiatement,
		de façon a ne pas perdre de temps (et donc en QoS) à redémarrer
		un autre serveur;		
	\item[licence] les entreprises fournissant les systèmes de virtualisation
		proposent des licences selon différents critères (eg. nombre de
		machines virtuelles lancées, utilisation de ressources (CPU, RAM, etc.));
	\item[plateforme] plusieurs plateformes de virtualisations sont disponibles
		(eg. Xen, VMWare, Citrix); une autre contrainte sur la
		répartition des machines virtuelles se pose alors, un serveur
		physique ne faisant tourner qu'un seul type de plateforme;
	\item[] $\ldots$
\end{description}
\subsection{Défis}
Afin de pouvoir répondre aux besoins exprimés par l'un des domaines
cité dans le paragraphe précédent, il est nécessaire de commencer
par formaliser le problème. En d'autres termes, donner une définition
mathématiques des contraintes impliquées par la problèmatique choisie,
et s'assurer qu'elles sont envisageables en pratique. Finalement, cette
représentation abstraite doit être implémentée sous forme de plugin
Java pour entropy~\cite{herm2009} un manager de clusters reposant sur
l'algorithme btrplace\footnote{\url{http://btrp.inria.fr/sandbox/about.html}}.

\subsection{Objectifs}
Actuellement, les trois derniers points cités ne sont pas forcément
formalisé/implémenté sous une forme satisfaisante. Le projet consiste
donc à choisir l'un de ces domaines et à l'ammener vers une forme
satisfaisante.

Le dernier point est celui sur lequel se porte ce projet.
\subsection{Scénarios}
XXX diagramme.
% \begin{figure}[!ht]
%         \centering
%         \includegraphics[scale=.3]{imgs/cloud.png}
% 	\caption{\label{usecase} Exemple de changement de système de virtualisation}
% \end{figure}
\begin{verbatim}
    |
    |               _____________
    |              |  VM2 (Xen)  |
    |              |_____________|
    |              |  VM1 (Xen)  |
 N4 |______________|_____________|
    |   VM5 (Xen)                |
    |____________________________|

    |
 N3 |____________
    |  VM4 (VMW) |  offline
    |____________|

    |_____________
 N2 |             |
    |  VM3 (VMW)  | offline
    |_____________|

    |
    |               ______________
    |____________  |  VM4 (VMW)   |
    |  VM2 (Xen) |o|______________|
 N1 |____________|f|              |
    |  VM1 (Xen) |f|  VM3 (VMW)   |
    |____________|_|______________|

\end{verbatim}
Sur le diagramme ci dessus, les serveurs physiques $N_3$ et $N_2$ doivent êtres mis
hors-ligne pour des questions de maintenances, via la contrainte
\textit{offline($N_i$);}\footnote{\url{http://www-sop.inria.fr/members/Fabien.Hermenier/btrpcc/offline.html}}.

Comme aucun serveur VMWare n'est disponible, il est nécessaire de supprimer
un serveur Xen, capable d'accueillir $VM_4$ et $VM_3$, par exemple $N_1$.

\subsection{Critère de succès}
bis repetita: trouver un bon formalisme; définir et implémenter
les contraintes.
\section{État de l'art}
\subsection{Description générale}
\subsection{foo}
\subsection{bar}

\section{Méthodologie et planification}
\subsection{Stratégie générale}
bis repetita: trouver un bon formalisme; définir et implémenter
\subsection{Découpage en lots}
bis repetita: trouver un bon formalisme; définir et implémenter
\subsection{Plannification}
gantt
\subsection{Livrables associés au projet}
\begin{table}
\centering
\begin{tabular}{c|c|c|c|c}
	Id & Titre du livrable & Lot(s) & Nature & Date \\
	\hline
	\hline
	$D_0$ & Cahier des charges & $1$ & Document & $S_4$ \\
	\hline
	$D_1$ & Rapport de management & $1$ & Document & $S_{20}$ \\
	\hline
	$D_2$ & Plugin entropy & $1$ & Logiciel & $S_{20}$ \\
	\hline
	$D_3$ & Diaporama de présentation finale & $1$ & Document & $S_{20}$ \\
\end{tabular}
\end{table}
\subsection{Jalons}
\begin{table}
\centering
\begin{tabular}{c|c|c|c|c}
	Id & Jalon de fin de phase & Lot(s) & Date & Vérification \\
	\hline
	\hline
	$J_0$ & planification & $1$ & $S_4$ & $D_0$ \\
	\hline
	$J_1$ & formalisation & $1$ & $S_n$ & $D_1$ partiel \\
	\hline
	$J_2$ & implémentation & $1$ & $S_{n+k}$ & $D_2$, $D_1$ partiel \\
	\hline
	$J_3$ & projet & $1$ & $S_{20}$ & $D_1$, $D_2$, $D_3$ \\
\end{tabular}
\end{table}

\section{Description de la mise en œuvre du projet}
\subsection{Interdépendance des lots et tâches}
bis repetita: trouver un bon formalisme; définir et implémenter
\subsection{Description des lots}
bis repetita: trouver un bon formalisme; définir et implémenter
\subsection{Résumé de l'effort}
\subsection{Gestion du risque}

\section{Participants}
\subsection{Mathieu Bivert - CSSR}
\subsection{Fabien Hermenier - OASIS/INRIA}

\newpage
\selectlanguage{francais}
\bibliographystyle{alpha}
\bibliography{docs}

\end{document}

